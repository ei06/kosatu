\documentclass[11pt,autodetect-engine]{jsarticle}% autodetect-engine で pLaTeX / upLaTeX を自動判定

\begin{document}
\noindent {\Large {\bf 日常生活の社会経済史(谷本ゼミ)}} \\
\hfill  (2020年12月10日 経済学研究科D1五十嵐英梨香)\\
\noindent \rule{\textwidth}{0.2mm} 
輪読本:\\
{\large {\it 板垣邦子(1992)「昭和戦前・戦中期の農村生活:雑誌『家の光』にみる」}}\\
{\large {\it 坂口正彦(2019)「近代日本の『むら仕事』」}}\\

\part{坂口正彦(2019)「近代日本の『むら仕事』」}
{\bf 
\Large はじめに}
\large\begin{itemize}
  \item 「むら仕事」はほとんどの集落で実施され慣習として存在が知られていても,研究対象として焦点が当てられていない
  \item 先行研究:むら仕事の種類は基盤整備(農道・水路)が多い。無償かそれに近い賃金のため貧困層に負担であったという指摘の一方,有償労働であり村落住民にとって現金獲得機会であったという指摘も
  \item 本稿の意義:むら仕事の分析を通じて近代日本村落の経済的機能の検討
  \\・農家経営に村落がどれほど関与したか(農業生産過程の共同,貧困層)
  \\・政策の受け皿として村落がどれほど機能していたか
  \item 分析対象:滋賀県神崎郡栗見荘村乙女浜村落(米作地域)
  \\乙女浜区有文書『諸人夫控』等に記録された1911-1946年が対象
\end{itemize}

 \section{「むら仕事」の量と種類}
 
\begin{itemize}
  \item 基盤整備:乙女浜でもこの仕事が一番多く,水利に関する仕事が多数を占める。基盤整備の実施は農家経営の最繁忙期をずらしていた
  \item 農業生産過程:共同作業としてむら仕事でなされることは少ないが,1920年代に麦奴予防が行われ始め,1930年代を経て戦時に増加した
  \item 集荷・運搬・分配:苗の運搬や農家から集荷した供出米の村落外への運搬を行う一方,農家から供出米を出荷する局面は「むら仕事」に含まれない←村落は農家経営の全てに関与したわけではない
  \item 村役場・小学校:小学校運動会の運営,教員住宅の整備,村役場の杭打ち,門改修など
  \item 講演・講習への参加:「模範村」である八幡村に村落が編入されると行政村が計画した経済構成運動への参加がむら仕事に
  \item 時期による変化:1910年代は基盤整備が多く(前半は村役場・小学校関連,後半は集荷・運搬・分配関連がそれぞれ増加),1920年代に農業生産過程関連が出現。1930年代前半はむら仕事量が少ないが後半には量的なピークに。戦時期には労働力不足のため量が減少し村落の機能低下へ
\end{itemize}


\section{「むら仕事」の賃金と担い手}
 \begin{itemize}
  \item 乙女浜でのむら仕事は有償労働で,1920年代では近隣の工業賃金の4.5-6割・農業日雇い賃金の4-5.3割ほどの賃金→半分ほどは損得勘定抜きに実施
  \item 村落協議費によって経済階層を明らかにすると,協議費63-87本(24.26-36.54銭)の経済階層を中心にむら仕事が担われた→貧困層は相対的に負担も少なく得られる利益も(田畑が少ないため)少ない
  \item 他の村落内労働の出役も貧困層は少なく,乙女浜村落で貧困層の雇用機会を生み出す役割はなかった
  \item むら仕事は組(近隣のまとまり)単位で割り当てられ,さらに各組のむら仕事による受益によって負担が増減→組間の負担公平化と受益者負担,二つの原理によって規定
\end{itemize}

{\bf 
\Large おわりに}
\large\begin{itemize}
  \item 1920年代以降,農業生産過程共同化を通して村落は農家経営に関与
  \item むら仕事における貧困層の負担は軽減されており村落は彼らに他の就業機会を与えたとも言えるが,むら仕事自体は現金獲得機会として十分な額ではない
  \item むら仕事に政策関連のものが組み込まれており受け皿として作用したが,戦時の労働力不足によってむら仕事が減少し村落の機能が低下
\end{itemize}

\part{板垣邦子(1992)「昭和戦前・戦中期の農村生活:雑誌『家の光』にみる」}
\setcounter{section}{0}

{\bf 
\Large まえがき}
\large\begin{itemize}
  \item 『家の光』:産業組合を通じて配布され,農民家族全員を対象とした廉価な雑誌。農村生活の尺度に合わせた記事が特徴で,プラクティカルな面を追求した内容の時に部数が飛躍的に伸長
  \\→プラクティカルな記事を分析し『家の光』全体像をとらえる
  \item 衣食住その他生活問題に関する農村大衆の志向を抽出
\end{itemize}

\section{創刊期(大正14年5月-昭和4年12月号)}

 \subsection{序 }
  \begin{itemize}
  \item 貧しい農民の社会的経済的地位向上や社会改良を目指していた産業組合が産業組合知識の普及のために『家の光』創刊
  \item 内容は模索期で,大衆雑誌だが「おもしろく」はない。産業組合や反都会的な記事が目立ち,地方産業組合周辺の青年層が主な読者層
  \item 遅々として普及しなかったが,特約組合制度と家の光大会という大衆宣伝が後の発展への布石に
\end{itemize}

 \subsection{農村生活改善論 }
 
 \begin{itemize}
  \item 資本主義発達による都市の繁栄に対して農村は疲弊し,農村青年層は反都会的感情を抱きつつ農村生活に物足りなさを感じる
  \\→『家の光』は農村文化の建設を提唱
  \item この時期の生活改善論は抽象的だが基本的枠組みは,①反都市・反資本主義,②反因習的で農村近代化を追求
  \\→昭和4年以降,抽象的な「農村文化」から「生活改善」という言葉に代わり使用され実用的な雑誌として『家の光』が完成
\end{itemize}

 \subsection{婦人問題 }
 
 \begin{itemize}
  \item 農村の疲弊と生活水準の低下・停滞
  \\→伝統から脱却し生活の近代化を推進することが必要
  \item 農村婦人の教育の低さが生活のみじめさや婦人の地位が低いことに寄与
  \\→教育を通して農家の家計に対処できる主婦になり生活改善できる知識と技術を習得することや,社会活動に積極的参加し婦人の地位向上を促すこと求められた
  \item 家父長制の否定と家族員の自立,女性自身の自覚
\end{itemize}

 \subsection{衣食住・食生活 }
 
 \begin{itemize}
  \item 衣生活記事:都市家庭向きで趣味的で当時の農村家庭において実用性に乏しい(和装や廃物利用関係の記事が少ない)が,早くも婦人の仕事着を動きやすい洋装にすることを主張
  \item 料理記事:趣味的で農村食生活とはかけ離れていた
\end{itemize}

\section{発展期(昭和5年1月-昭和12年8月号) }

 \subsection{序 }
 
 \begin{itemize}
  \item 農業恐慌下,産業組合は経済厚生運動の中心的役割を担い,農村の流通機構も掌握
  \\→産業組合の組織網を生かした『家の光』の普及へ
  \item 梅山一郎を編集主任に迎え,農村向家庭雑誌として誌面を充実
  \\①初等教育終了程度の農村大衆,特に子供・女性を読者として設定
  \\②他の大衆雑誌のように特色あるシリーズ記事
  \\③農村生活記事に重点
  \\④編集面で読者の参加を積極的に求める
  \item 昭和10年度に100万部達成し,収入は急増
\end{itemize}

 \subsection{衣生活 }
 
 \begin{itemize}
  \item 当時の農村は貧しく,経済的・技術的理由により洋装化は進展しておらず,農業労働と家事労働をする主婦にとって衣類管理は大きな負担
  \item 『家の光』は衣類について簡単な仕立方や廃物利用法・自給の勧めに関する記事を出し,単純化・能率化・衛生的な衣生活を目指す
  \item 子供服は作業服や制服などを通して洋装化が進展したが,婦人向作業服改善への主体的な取り組みは少ない
\end{itemize}

 \subsection{食生活 }
 
 \begin{itemize}
  \item 当時の農村は主食物から熱源の大半を得ていて副食物は乏しい
  \item 『家の光』は栄養知識に基づいた「新農村料理」の確立や,向上的・共同的自給生活を主張
  \item 食生活改善には婦人の役割が重要だが男性に隷属的地位にあったことや経済的困窮によって,向上を志向しつつも食生活は切り詰められ粗末であった
\end{itemize}

 \subsection{住宅 }
 
 \begin{itemize}
  \item 農村住宅は作業の場としての意義が縮小し,能率・衛生に配慮した生活(家庭)重視の方向へ(子供部屋を除き個室化にはならず)
  \item 住宅改造,整理整頓や清潔を目指したり,明るく衛生的な台所や効率的な竃への改善を『家の光』は主張
  \item 改善費用は高かったが,段階的な改装や部落ぐるみでの改善(共同副業など)によって生活改善実現へ
\end{itemize}


 \subsection{時間 }
 
 \begin{itemize}
  \item 農業生産構造の根本的な改善が望めず,農民は勤勉努力主義によって収入増加を図る→時間励行
  \item 規則的・計画的に時間を使用することで生活に余裕や個人の余暇確保へ(村人や家族一同での余暇ではなく個人のものを要求)
\end{itemize}

 \subsection{儀礼 }
 
 \begin{itemize}
  \item 社交儀礼費の中でも結婚費は節約の矢面に立たされ,結婚式の合理化を目指す
  \item 結婚相手の理想像として男性は合理的な家計処理能力を有している女性を挙げ,結婚には本人の意思の尊重を望む。女性側も足入れ婚などの因習に反発
  \item 農村女性は婚姻費が少なくてすみ家や男性優位に囚われずにすむ都会の男性との結婚を志向するように
\end{itemize}

 \subsection{保険・医療 }
 
 \begin{itemize}
  \item 衛生状態は悪く,医療費は農家負債の原因に
  \\→病気予防のため衛生観念を養うため,啓蒙・指導,母子保健,医療利用組合,健康法などの記事を『家の光』は掲載
  \item 国策として建民健兵策が展開すると,『家の光』は農村救済策から建民健兵へ「病人を一人でも少なくする運動」を変化させた
\end{itemize}

 \subsection{娯楽・文化 }
 
 \begin{itemize}
  \item 困窮による節約は時に農村の娯楽である伝統的な行事や風習を切り捨てた
  \\→『家の光』は読切や漫画などの娯楽そのものを掲載し大衆娯楽雑誌としての性格を確立
  \item 金をかけずに楽しめる演劇会や運動会などの「共同」娯楽を奨励
\end{itemize}

 \subsection{婦人問題 }
 
 \begin{itemize}
  \item 『家の光』は主人の協力者として対等な「主婦」像を提示
  \\→家政合理化のため自給品や廃物利用に知恵を絞り,時間と労力も節約し家計も管理する主婦像(消費生活面における言及が中心)
  \item 深刻化する経済恐慌の中で婦人会活動が生み出される
  \\→広い意味での生活改善を目指すものの目前の経済更生が第一義。村・農会・産業組合の指導下で非自主的な活動
\end{itemize}

 \subsection{農村生活改善論 }
 
 \begin{itemize}
  \item 『家の光』が提起した生活改善構想は①反都会・反資本主義②反因習
  \\→生活改善の費用を生み出すための自給自足や家族・部落の共同を強調した農村風のモダニズム
  \\→生活改善へ意欲のある自作・小作層や青年・婦人の広汎な支持を得る
  \item 『家の光』が推奨する消費の抑制・禁止,労働強化策の理想と現実の農村には落差があり生活改善は修正されたが,以後着実に『家の光』の理想よりも都市的モダニズムが浸透
\end{itemize}

\section{戦中期(昭和12年9月-昭和20年4・5月号) }

 \subsection{序 }
 
 \begin{itemize}
  \item 戦時体制によって産業組合は統制団体へ統合され自由な出版活動が制限されていき,『家の光』の内容も生活改善ではなく戦争や食糧生産力拡充を扱う
\end{itemize}

 \subsection{衣生活 }
 
 \begin{itemize}
  \item 衣料物資は不足し,廃物利用が受動的にせざるを得なくなったり耐久性のない繊維の使用から婦人の負担が増加
  \item 国民服や婦人標準服が制定され,『家の光』は農村婦人作業着の改善に力を入れた(この改善は婦人自身より動員させたい側の意向が先行)
\end{itemize}

 \subsection{食生活 }
 
 \begin{itemize}
  \item 農民の健康状態が悪化し農村保険問題に関心が払われ,食生活改善のため『家の光』は栄養指導をし共同炊事を促す
  \item 都市部よりも農村の食生活の質は高かったと考えられるが割当制の配給制度や労働人口の流出によって農村は大打撃
\end{itemize}

 \subsection{保険・医療}
 
 \begin{itemize}
  \item 戦時体制が老幼婦女の動員と労働強化を求め,『家の光』は「建民健兵の村づくり」を強調。農村保健問題解決は農民側も強く望む
  \item 母性・乳幼児に関する健康問題は婦人の農業労働従事によって規定されており,医師・保健婦の配置のみでは解決せず,戦争末期には保健状態は荒廃
\end{itemize}

 \subsection{娯楽・文化 }
 
 \begin{itemize}
  \item 『家の光』は国策伝達雑誌化し,娯楽記事も娯楽を通した国策協力を呼びかけるもの(例・戦争関連の読物,体位向上を目的としたスポーツなど)
  \item 大政翼賛会は生産能率向上のために娯楽に注目し文化事業を活発化
  \item 農村青年は共同娯楽より個人的・趣味的な新しい娯楽や文化を望んだが保守的な親に干渉されがち
\end{itemize}

 \subsection{婦人問題 }
 
 \begin{itemize}
  \item 銃後を守る婦人の役割が重視される
  \\①家庭生活を管理する主婦(共同して,科学的に,無駄を省く)
  \\②忍従,献身,優しさと強さの象徴としての「母」→人的資源確保を狙う人口政策
  \\③食糧生産を担う農業生産者(労働力確保のための共同炊事,託児所など)
  \item 国家は女性に家庭にありながら国家・社会に貢献するよう求めたが,当の女性は家に従順な「家庭的個人主義」
  \item 生活様式の変化や結婚先に農村を敬遠する背景もあり,因習を打破し明朗な農村を建設しようと,農民が主体的に自分の生活に余裕を生み豊かにしたいと(婦人に限って言えば男性や貧しさからの解放を求める)「生活改善」へ意欲的
  \item 農村の婦人団体は具体的プランの不足や主体的な自覚を欠く
\end{itemize}

 \subsection{農村生活改善論 }
 
 \begin{itemize}
  \item 生活記事は一時減少したものの昭和16年頃から農業・生活に関する実用記事が増加←農林当局の指示の下に展開され押しつけ的
  \item 昭和16年6月号の「食糧増産推進の頁」新設以降,国策伝達誌へと転換
  生活改善は,生活向上を願う農村大衆の主体的営みから,個人より国家を優先する受動的なものへ
\end{itemize}


\noindent \rule{\textwidth}{0.2mm} 
{\bf 
\Large コメント・論点}
\begin{itemize}
  \item (坂口)行政村への編入によってむら仕事の内容が変化した記載があるが(59頁),これはむら仕事が同時代であっても地域の属性によって差異が生じたことを示唆する。この研究の代表性はどの程度あると考えられるか。
  \item (板垣)発展期において,農村(特に青年層)の理想と『家の光』が奨励した生活改善には微妙なズレが存在していたとあるが,それでも『家の光』が支持を得て発行部数を伸ばしていた理由は農村に受け入れられる他の大衆誌や知識の仕入れ先がなかったからなのか。
  \item (板垣)戦時期になると国策伝達誌と化した『家の光』と農村大衆の求めているものとの乖離が大きくなるような記述が目立つが,『家の光』を分析したことで農村大衆の志向を抽出できたといえるのか。
  \item (全体)むら仕事や産業組合を通じて国家は村(農村大衆)に関与しており,その分析を通じて当時の生活について大まかに把握することができ大変興味深かった。しかしそれは当時の人々の一つの側面や国家・村落・組合(監督する側)から見た様子にすぎず,改めて「一般人」の生活について研究する難しさを感じた。
\end{itemize}

\end{document}